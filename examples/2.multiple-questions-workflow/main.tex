\documentclass{article}

\usepackage{amsmath}
\usepackage{juliaunit}
\usepackage{siunitx}   % Already loaded by juliaunit, but useful to get code completion.

% The actual code will be ran from the pythontex-files-main folder. 
% So code.jl is in the parent folder (which is the same as the .tex file).
\begin{jlcode}
    include("../code.jl");
\end{jlcode}

\begin{document}
    \section*{Question 1}
    Now I have access to all my variables in \texttt{code.jl}, where I put mostly my functions and other stuff. The actual variables for this questions goes anywhere here.
    \begin{jlcode}
        T = 298.0 * u"K";
        P = 1u"bar";
        n = 1u"mol";
        V = n * R * T / P; # The R constant comes from code.jl!
    \end{jlcode}
    At the end,
    \[ PV = nRT \iff V = \frac{nRT}{P} \implies V = \jlunit[2]{V |> u"L"}\;. \]
    
    \section*{Question 2}
    \begin{jlcode}
        MM = 28.0134u"g/mol"; # Molar Mass
        m = 38u"g"; # Mass
        T = -100u"°C" |> u"K"; # -100°C, but in Kelvin :)
        P = 1u"atm";
        n = m / MM;
        V = n * R * T / P; # The R constant comes from code.jl!
    \end{jlcode}
    We have \SI{38}{\gram} of N$_2$. Considering the ideal behavior, the volume is, at $ \jlunit{T} $,
    \[ PV = nRT \iff V = \frac{nRT}{P} \implies V = \jlunit[2]{V |> u"L"}\;. \]
\end{document}